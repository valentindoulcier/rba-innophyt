\documentclass[twoside]{EPURapport}
%\usepackage{listings}
\usepackage[utf8]{inputenc}

%\renewcommand{\lstlistlistingname}{Liste des codes}
%\renewcommand{\lstlistingname}{Code}

%\addextratables{%
%	\lstlistoflistings
%}

%\swapAuthorsAndSupervisors



\usepackage[utf8]{inputenc}

\thedocument{Projet d'option web}{Évolution d'un classifieur multimédia, interactif et en ligne}{Projet d'option web - Évolution d'un classifieur multimédia, interactif et en ligne}

\grade{Département Informatique\\ 5\ieme{} année\\ 2012 - 2013}

\authors{%
	\category{Étudiants}{%
		\name{Adrien BATAILLE} \mail{adrien.bataille@etu.univ-tours.fr}
		\name{Valentin DOULCIER} \mail{valentin.doulcier@etu.univ-tours.fr}
	}
	\details{DI5 2012 - 2013}
}

\supervisors{%
	\category{Encadrants}{%
		\name{Gilles VENTURINI} \mail{gilles.venturini@univ-tours.fr}
		\name{Fabienne DALINO} \mail{fabienne.dalino@gmail.com}
	}
	\details{Université François-Rabelais, Tours}
}

\abstracts{Description en français}
{Mots clés en Français}
{Description en anglais}
{Mots clés en anglais}

\begin{document}


\chapter{Introduction}

	\section{Contexte}
	
	Ce projet s'inscrit dans le cadre de notre formation d'ingénieur en informatique. Toutefois, il fait suite à de nombreux projets...
	
	\section{Objectifs}
	
	Réaliser un site qui ressemble à l'application tablette en intégrant le quizz déjà existant mais en utilisant le nouvel arbre.
	
	\section{Contraintes technologiques}

\chapter{Architecture}

	\section{Arborescence}
	
	Modification de la base fournie
	
		\subsection{•}
		
		Détailler éléments par éléments ...
	
	\section{Librairies}
		
		\subsection{Google Map}
		
		\subsection{jQuery}
		
		\subsection{jQuery UI}
	
		\subsection{jqPlot}
		
		\subsection{Bootstrap}
		
		\subsection{Shadowbox}

\chapter{Analyse Critique}

	\section{Du projet}
	
	Dire ce qui est bien, cool dans se qu'on a fait et les points qui peuvent être améliorés
	
	\section{Améliorations}
	
	Donner les points à améliorer pour les prochains projets

\chapter{Conclusion}

\annexes

\end{document}

